\documentclass[dvipdfmx]{jsarticle}
\begin{document}

\title{Hello tex}
\author{knzk}
\maketitle

\section{はじめに}

この文書は、ごく基本的なレポートや論文の例を示すものです。
実際にこのソースを入力してタイプセット(コンパイル)し、
タイトル、著者名、本文、見出し、箇条書きがどのように表示されるかを
確認してみましょう.

できたああああああああああ

\section{みだし}

この文書の先頭にはタイトル、著者名、日付が出力されています。
特定の日付を指定することも可能です

そしてセクションの見出しが出力されています。
セクションの番号は自動的につきます。

\section{箇条書き}

以下は箇条書きの例です。これは番号を振らない箇条書きです。

\begin{itemize}
  \item ちゃお
  \item りぼん
  \item なかよし
\end{itemize}

これは番号を振る箇条書きです。
\begin{enumerate}
  \item 富士
  \item 鷹
  \item なすび
\end{enumerate}

\section{終わりに}

これは一段組の例ですが、に段組みに変更する事もできます。
解説文を呼んで、このソースをいろいろと変更してみましょう。

\end{document}