\documentclass[dvipdfmx]{jsarticle}

\usepackage{ascmac}
\usepackage{url}
\usepackage[dvipdfmx]{hyperref}
\usepackage{pxjahyper}
\usepackage[dvipdfmx]{graphicx}
\usepackage{float}
\usepackage{listings,jlisting}

\hypersetup{
  colorlinks=true,
  urlcolor=cyan,
  linkcolor=black
}

\lstset{
  basicstyle={\ttfamily},
  identifierstyle={\small},
  commentstyle={\smallitshape},
  keywordstyle={\small\bfseries},
  ndkeywordstyle={\small},
  stringstyle={\small\ttfamily},
  frame={tb},
  breaklines=true,
  columns=[l]{fullflexible},
  numbers=left,
  xrightmargin=0zw,
  xleftmargin=3zw,
  numberstyle={\scriptsize},
  stepnumber=1,
  numbersep=1zw,
  lineskip=-0.5ex
}


\begin{document}

\section{実験目的・課題}
以下の3つの課題を行う。
\begin{itemize}
  \item 課題1
        \begin{itemize}
          \item 数値微分の方法を理解する
          \item $sin(x)$を微分するプログラムを作成する
          \item 分割数を変更し精度が変化することを確認する
        \end{itemize}
  \item 課題2
        \begin{itemize}
          \item 数値積分の方法を理解する
          \item 円の面積を求めるプログラムを作成する
          \item 分割数を変更し精度が変化することを確認する
        \end{itemize}
  \item 課題3
        \begin{itemize}
          \item 上記以外の微分法・積分法について調査しプログラムを実装する
        \end{itemize}
\end{itemize}

\section{実装方法}

\subsection{課題1}

式\ref{bibun1}によって関数$f(x)$の微分は定義される。
\begin{equation}
  f'(x) = \lim_{h \to 0} \frac{f(x+h)-f(h)}{h}
  \label{bibun1}
\end{equation}
しかし、計算機では極限の計算はできない。そのため、
hに微小量を設定して近似して解を得る。

微分を計算する差分のとり方は3つあり、それぞれ
前方差分(式\ref{forward})、後方差分(式\ref{backward})、中心差分(式\ref{central})である。
\begin{equation}
  f'(x) \simeq \frac{f(x+h)-f(x)}{h}
  \label{forward}
\end{equation}
\begin{equation}
  f'(x) \simeq \frac{f(x)-f(x-h)}{h}
  \label{backward}
\end{equation}
\begin{equation}
  f'(x) \simeq \frac{f(x+h)-f(x-h)}{2h}
  \label{central}
\end{equation}


\subsection{課題2}

関数$f(x)$の値域aからbまでの定積分は式\ref{integral}で計算することが出来る。
\begin{equation}
  \int_a^b f(x) dx = \lim_{\Delta \to 0} \sum_{x=a}^{b} f(x) \Delta
  \label{integral}
\end{equation}
しかし、計算機では極限の計算が出来ないため、
$\Delta$を微小量として長方形の面積の和を計算し近似的な値を得る。

\subsubsection{台形公式}
定積分$\int_a^b f(x)dx$の値はxy座標平面で$y=f(x)$とx軸の区間$[a,b]$で
囲まれた面積になる。
よって近似値は式\ref{daikei1}によって計算できる。
\begin{equation}
  \int_a^b f(x)dx \simeq (b-a)\frac{f(a)+f(b)}{2}
  \label{daikei1}
\end{equation}

この式では、曲線$y=f(x)$が直線から離れているほど精度が悪くなる。
そこで、積分区間$[a,b]$をn個の区間$[a_0,a_1],[a_1,a_2],...,[a_{n-1},a_n] (a_0=a,a_n=b)$
に分割し、それぞれで台形公式を適用しその和で面積を近似計算する(式\ref{daikei2})。
\begin{equation}
  \int_a^b f(x) \simeq \sum_{k=1}^{n} (a_k-a_{k-1})\frac{f(a_{k-1})+f(a_k)}{2}
  \label{daikei2}
\end{equation}

\subsubsection{シンプソンの公式}
シンプソンの公式は$f(x)$を二次関数$g(x)$で近似することで導かれる。
$g(x)$は$f(x)$の点$a,m,b~(m=(a+b)/2)$におけるラグランジュ補完によって次の多項式(式\ref{sim1})になる。
\begin{equation}
  g(x) = f(a)\frac{(x-m)(x-b)}{(a-m)(a-b)}+f(m)\frac{(x-a)(x-b)}{(m-a)(m-b)}+f(b)\frac{(x-a)(x-m)}{(b-a)(b-m)}
  \label{sim1}
\end{equation}

この多項式を$[a,b]$で積分すると式\ref{sim2}(シンプソンの公式)が得られる。
\begin{eqnarray}
  \int_a^b f(x)dx ~&\simeq&~ \int_a^b g(x)dx \nonumber \\
  &=&~ \frac{b-a}{6} \left[f(a)+4f(\frac{a+b}{2})+f(b)\right]
  \label{sim2}
\end{eqnarray}

\subsection{課題3}
ガウスの数値積分公式を用いて数値積分を行う。
\subsubsection{ガウス求積}
ガウスの数値積分公式とは、nを正の整数とし、$f(x)$の[-1,1]の積分を
式\ref{gauss1}の形で近似する公式のことである。
\begin{equation}
  \int_{-1}^{1} \simeq \sum_{i=1}^{n} w_i f(x_i)
  \label{gauss1}
\end{equation}
ここで、$x_i$は積分点またはガウス点と呼ばれる[-1,1]内のn個の点であり、
$w_i$は重みと呼ばれるn個の実数である。

n次のルジャンドル多項式の[-1,1]内にあるn個の零点を積分点として選び、
$w_i$を適切に選ぶと、$f(x)$が$2n-1$次以下の多項式であれば式\ref{gauss1}が
厳密に成立する。
この方法をn次のガウス・ルジャンドル公式と呼び、通常は
ガウス求積と言えばこの方法を指す。

$f(x)$が$2n-1$次を超える多項式関数または、多項式関数ではない場合、
式\ref{gauss1}は厳密には成立しないが、$f(x)$が$2n-1$次以下の多項式関数
で精度よく近似できる場合には$f(x)$に式\ref{gauss1}を適用することにより
精度よく定積分値を得ることが出来る。

式\ref{gauss1}の積分を区間$[a,b]$に適用したいときは
新たな変数tでの原点を[a,b]の中点とし、1目盛りの幅を$(b-a)/2$とすると、
\begin{eqnarray}
  t &=& \frac{a+b}{2} - \frac{a-b}{2}x \nonumber \\
  dt &=& \frac{b-a}{2}dx \nonumber
\end{eqnarray}
から、
\begin{eqnarray}
  \int_a^b f(x)dx ~&=&~ \int_{-1}^{1} f(t)dt \nonumber\\
  &=&~ \frac{b-a}{2} \int_{-1}^{1} f(\frac{a+b}{2} + \frac{b-a}{2}x)dx \\
  &=&~ \frac{b-a}{2} \sum_{i=1}^{n} w_i f(\frac{b-a}{2}x_i + \frac{a+b}{2})
  \label{gauss2}
\end{eqnarray}
と変形できる。

\subsubsection{ガウス求積法の分点と重み}
ガウス求積の分点はルジャンドル多項式$P_n(z)$、$P_n(z)$は
\begin{eqnarray}
  P_0(z) ~&=&~ 1 \nonumber\\
  P_1(z) ~&=&~ z \nonumber\\
  P_k(z) ~&=&~ \frac{2k-1}{k}zP_{k-1}(z) - \frac{k-1}{k}P_{k-2}(z) ~,~2\leq k
\end{eqnarray}

\section{結果と考察}

\begin{thebibliography}{10}
  \bibitem{1} 台形公式
  \url{https://ja.wikipedia.org/wiki/%E5%8F%B0%E5%BD%A2%E5%85%AC%E5%BC%8F}
  \bibitem{2} シンプソンの公式
  \url{https://ja.wikipedia.org/wiki/%E3%82%B7%E3%83%B3%E3%83%97%E3%82%BD%E3%83%B3%E3%81%AE%E5%85%AC%E5%BC%8F}
  \bibitem{3} ガウス求積
  \url{https://ja.wikipedia.org/wiki/%E3%82%AC%E3%82%A6%E3%82%B9%E6%B1%82%E7%A9%8D}
  \bibitem{数値解析 加古富志雄 令和元年 5 月 27 日}
  \url{https://www.ics.nara-wu.ac.jp/~kako/teaching/na/chap6.pdf}

\end{thebibliography}
\end{document}