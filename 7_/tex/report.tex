\documentclass[dvipdfmx]{jsarticle}

\usepackage{ascmac}
\usepackage{url}
\usepackage[dvipdfmx]{hyperref}
\usepackage{pxjahyper}
\usepackage[dvipdfmx]{graphicx}
\usepackage{float}
\usepackage{listings,jlisting}

\hypersetup{
  colorlinks=true,
  urlcolor=cyan,
  linkcolor=black
}

\lstset{
  basicstyle={\ttfamily},
  identifierstyle={\small},
  commentstyle={\smallitshape},
  keywordstyle={\small\bfseries},
  ndkeywordstyle={\small},
  stringstyle={\small\ttfamily},
  frame={tb},
  breaklines=true,
  columns=[l]{fullflexible},
  numbers=left,
  xrightmargin=0zw,
  xleftmargin=3zw,
  numberstyle={\scriptsize},
  stepnumber=1,
  numbersep=1zw,
  lineskip=-0.5ex
}


\begin{document}

\section{実験目的・課題}
以下の3つの課題を行う。
\begin{itemize}
  \item 課題1
        \begin{itemize}
          \item 数値微分の方法を理解する
          \item $sin(x)$を微分するプログラムを作成する
          \item 分割数を変更し精度が変化することを確認する
        \end{itemize}
  \item 課題2
        \begin{itemize}
          \item 数値積分の方法を理解する
          \item 円の面積を求めるプログラムを作成する
          \item 分割数を変更し精度が変化することを確認する
        \end{itemize}
  \item 課題3
        \begin{itemize}
          \item 上記以外の微分法・積分法について調査しプログラムを実装する
        \end{itemize}
\end{itemize}

\section{実装方法}

\subsection{課題1}

式\ref{bibun1}によって関数$f(x)$の微分は定義される。
\begin{equation}
  f'(x) = \lim_{h \to 0} \frac{f(x+h)-f(h)}{h}
  \label{bibun1}
\end{equation}
しかし、計算機では極限の計算はできない。そのため、
hに微小量を設定して近似して解を得る。

微分を計算する差分のとり方は3つあり、それぞれ
前方差分(\ref{forward})、後方差分(\ref{backward})、中心差分(\ref{central})である。
\begin{equation}
  f'(x) \simeq \frac{f(x+h)-f(x)}{h}
  \label{forward}
\end{equation}
\begin{equation}
  f'(x) \simeq \frac{f(x)-f(x-h)}{h}
  \label{backward}
\end{equation}
\begin{equation}
  f'(x) \simeq \frac{f(x+h)-f(x-h)}{2h}
  \label{central}
\end{equation}


\subsection{課題2}

\begin{equation}
  \int_a^b f(x) dx = \lim_{\delta \to 0} \sum_{x=a}^{b} f(x) \delta
\end{equation}

\subsection{課題3}

\section{結果と考察}

\begin{thebibliography}{10}
  \bibitem{1}

  \url{}
\end{thebibliography}
\end{document}