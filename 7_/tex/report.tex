\documentclass[dvipdfmx]{jsarticle}

\usepackage{ascmac}
\usepackage{url}
\usepackage[dvipdfmx]{hyperref}
\usepackage{pxjahyper}
\usepackage[dvipdfmx]{graphicx}
\usepackage{float}
\usepackage{listings,jlisting}

\hypersetup{
  colorlinks=true,
  urlcolor=cyan,
  linkcolor=black
}

\lstset{
  basicstyle={\ttfamily},
  identifierstyle={\small},
  commentstyle={\smallitshape},
  keywordstyle={\small\bfseries},
  ndkeywordstyle={\small},
  stringstyle={\small\ttfamily},
  frame={tb},
  breaklines=true,
  columns=[l]{fullflexible},
  numbers=left,
  xrightmargin=0zw,
  xleftmargin=3zw,
  numberstyle={\scriptsize},
  stepnumber=1,
  numbersep=1zw,
  lineskip=-0.5ex
}


\begin{document}

\section{実験目的・課題}
以下の3つの課題を行う。
\begin{itemize}
  \item 課題1
  \begin{itemize}
    \item 数値微分の方法を理解する
    \item $sin(x)$を微分するプログラムを作成する
    \item 分割数を変更し精度が変化することを確認する
  \end{itemize}
  \item 課題2
  \begin{itemize}
    \item 数値積分の方法を理解する
    \item 円の面積を求めるプログラムを作成する
    \item 分割数を変更し精度が変化することを確認する
  \end{itemize}
  \item 課題3
  \begin{itemize}
    \item 上記以外の微分法・積分法について調査しプログラムを実装する
  \end{itemize}
\end{itemize}

\section{実装方法}

\subsection{課題1}

\begin{equation}
  f'(x) =  \frac{f(x+h)-f(h)}{h}
  %\label{bibun1}
\end{equation}

\subsection{課題2}
\subsection{課題3}

\section{結果と考察}

\begin{thebibliography}{10}
  \bibitem{1} 
  
  \url{}
\end{thebibliography}
\end{document}