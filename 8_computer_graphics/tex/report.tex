\documentclass[dvipdfmx]{jsarticle}

\usepackage{ascmac}
\usepackage{url}
\usepackage[dvipdfmx]{hyperref}
\usepackage{pxjahyper}
\usepackage[dvipdfmx]{graphicx}
\usepackage{float}
\usepackage{listings,jlisting}

\hypersetup{
  colorlinks=true,
  urlcolor=cyan,
  linkcolor=black
}

\lstset{
  basicstyle={\ttfamily},
  identifierstyle={\small},
  commentstyle={\smallitshape},
  keywordstyle={\small\bfseries},
  ndkeywordstyle={\small},
  stringstyle={\small\ttfamily},
  frame={tb},
  breaklines=true,
  columns=[l]{fullflexible},
  numbers=left,
  xrightmargin=0zw,
  xleftmargin=3zw,
  numberstyle={\scriptsize},
  stepnumber=1,
  numbersep=1zw,
  lineskip=-0.5ex
}


\begin{document}

\section{実験目的・課題}

次の画像(図\ref{fig:tapu})について、以下の3つの処理を行う。

\begin{itemize}
  \item 画像に対して輪郭検出を行う
  \item 画像に対してフィルタ処理を行う
  \item 画像に対して自動検出によるマスク処理を行う
\end{itemize}

\begin{figure}[H]
  \centering
  \includegraphics[width=0.7\hsize]{../pic/tapu.png}
  \caption{画像処理を行う画像}
  \label{fig:tapu}
\end{figure}


\section{実装方法}

\subsection{画像の2値化}

画像の2値化とは、画素の特定の要素において閾値を与え、
0か1に分類することである。
二値化された画像は、図\ref{fig:binarization}のように白黒画像なる。
\begin{figure}[H]
  \centering
  \includegraphics[width=0.7\hsize]{../pic/binarization.jpg}
  \caption{大津の2値化を適用した図\ref{fig:tapu}}
  \label{fig:binarization}
\end{figure}


\subsection{輪郭検出}



\subsection{フィルタ処理}



\subsection{自動検出によるマスク処理}




\section{結果と考察}



\begin{thebibliography}{10}
  \bibitem{1} 
  
  \url{}
\end{thebibliography}
\end{document}