\documentclass[dvipdfmx]{jsarticle}

\usepackage{ascmac}
\usepackage{url}
\usepackage[dvipdfmx]{hyperref}
\usepackage{pxjahyper}
\usepackage[dvipdfmx]{graphicx}
\usepackage{float}
\usepackage{listings,jlisting}

\hypersetup{
  colorlinks=true,
  urlcolor=cyan,
  linkcolor=black
}

\lstset{
  basicstyle={\ttfamily},
  identifierstyle={\small},
  commentstyle={\smallitshape},
  keywordstyle={\small\bfseries},
  ndkeywordstyle={\small},
  stringstyle={\small\ttfamily},
  frame={tb},
  breaklines=true,
  columns=[l]{fullflexible},
  numbers=left,
  xrightmargin=0zw,
  xleftmargin=3zw,
  numberstyle={\scriptsize},
  stepnumber=1,
  numbersep=1zw,
  lineskip=-0.5ex
}


\begin{document}

\section{実験目的・課題}

Socket通信を行いクライアントと通信し、
クライアントから入力された成績情報を
記録するサーバプログラムを作成する。

作成するサーバーの要件
\begin{itemize}
  \item サーバープログラムとクライアントプログラムをわけて作成
  \item 接続したクライアントは何度でもメッセージのやり取りが出来る
  \item 新しく成績情報と登録できる
  \item 登録された成績情報を記録できる
  \item 今までに登録された成績情報を閲覧できる
  \item 名前か番号で検索できる
  \item 入力にエラーがある場合は通知できる
  \item 重複登録はできない
\end{itemize}


\section{実装方法}
Javaで複数のプログラム間でデータのやり取りを行うときは、
Socket通信というものを用いて通信を行う。

ここに、2つのプログラムがあり、この2つで通信を行いたいとき、
一方を「サーバー」もう一方を「クライアント」として話をする。
サーバーとクライアントは最初、何のつながりも持っていない。
ここで、サーバー側がServerSocketクラスをインスタンス化し、
その際に受け取ったポート番号の監視を始める。
次に、クライアント側でSocketクラスにサーバー名とポート番号を渡し
インスタンス化する際にサーバーに接続要求が送られる。
サーバー側ではこのSocketによる接続要求をacceptメソッドによって
受け取り、これで2つのプログラム間の接続が完了する。

\subsection{全体の方針}
今回の成績情報サーバーの作成では、サーバーとクライアントが交互に
入出力を行うことにする。
サーバーからクライアントにメッセージを送信、
それを受け取ったクライアントがそのメッセージを標準出力に出力、
それを見たユーザーがそれに対する応答をキーボードから入力する。
その入力をクライアントよりサーバーに送信し、
その入力を受けてサーバーで処理を行う。
このまとまりを処理のひとまとまりとして扱い、
これを繰り返すことで成績情報サーバーを作成する。
この処理の例を図\ref{example1}に示す。
\begin{figure}[H]
  \centering
  \includegraphics[width=0.7\hsize]{../pic/1.png}
  \caption{サーバーとクライアントの入出力例}
  \label{example1}
\end{figure}

クライアントプログラムでは、処理のひとまとまりを順に処理すればよいので
ソケット入力、標準出力、標準入力、ソケット出力を順に1回だけ行うコードを
whileで囲い何度もループさせればよい。

サーバープログラムでは、サーバーの要件に従い、
成績情報の登録、検索、閲覧ができるものを作成する。

\subsection{情報の記録}
今回作成するプログラムは登録された情報を外部ファイルに書き込み保存し、
次回サーバー起動時にそれを読み込み登録済み情報として保持することが出来るようにする。
成績情報が書かれたファイルからの入力は「実験4 ファイルからの入力と正規表現」
で作成したプログラムのものを改変して使用する。
出力ファイルへの書き込みはプログラム終了時に行い、
Number,Name,Scoreをタブ区切りで出力する。

プログラム中でseisekiクラスのデータの保持にはArrayListではなく
TreeMapを使用する。
TreeMap K,Vは型Kのキーによって型Vの値にアクセスすることができるデータ構造である。
型Kによって比較する赤黒木によって実装されており、
要素数Nのとき、要素の検索、挿入、削除がlog(n)時間で行うことができるため、
検索の面でArrayListより優れていると判断し、これを採用した。

また、成績情報の入力において番号がバラバラに入力される可能性があるが、
TreeMapでデータを保存しておけばプログラムの最後に出力するときにキーの
昇順で成績情報を得られるので、
データの閲覧時にわかりやすさが向上することが見込める。

\subsection{登録}
成績情報の登録にはNumber,Name,Score1,2,3,4が必要である。
今回はこれを一度に入力してもらうのではなく、それぞれについて
クエリを投げてそれに回答してもらう形式にする。
こうしなかった場合、例えばNumberとNameの区切り文字を
どうするかという事も決めなければならなくなるうえ、
入力にミスが含まれる可能性が高くなると考えられる。
こうしてNumberを表す文字列、Nameを表す文字列、...を得られる。
次にそれに区切り文字を付け一つの文字列になるよう連結する。
このようにして得られた文字列が正規表現に
マッチすれば入力をseisekiクラスに変換できる。



\subsection{検索}
\subsection{閲覧}


\section{結果と考察}

\begin{thebibliography}{10}
  \bibitem{1} Javaプログラムにおける通信のしくみを理解する

  \url{https://crew-lab.sfc.keio.ac.jp/lectures/2000s_mmb/JavaLectures/Lecture8/Lec8-1.html}
\end{thebibliography}
\end{document}