\documentclass[dvipdfmx]{jsarticle}

\usepackage{ascmac}
\usepackage{url}
\usepackage[dvipdfmx]{hyperref}
\usepackage{pxjahyper}
\usepackage[dvipdfmx]{graphicx}
\usepackage{float}
\usepackage{listings,jlisting}

\hypersetup{
  colorlinks=true,
  urlcolor=cyan,
  linkcolor=black
}

\lstset{
  basicstyle={\ttfamily},
  identifierstyle={\small},
  commentstyle={\smallitshape},
  keywordstyle={\small\bfseries},
  ndkeywordstyle={\small},
  stringstyle={\small\ttfamily},
  frame={tb},
  breaklines=true,
  columns=[l]{fullflexible},
  numbers=left,
  xrightmargin=0zw,
  xleftmargin=3zw,
  numberstyle={\scriptsize},
  stepnumber=1,
  numbersep=1zw,
  lineskip=-0.5ex
}


\begin{document}

\section{実験目的・課題}

Socket通信を行いクライアントと通信し、
クライアントから入力された成績情報を
記録するサーバプログラムを作成する。

作成するサーバーの要件
\begin{itemize}
  \item サーバープログラムとクライアントプログラムをわけて作成
  \item 接続したクライアントは何度でもメッセージのやり取りが出来る
  \item 入力された成績情報を記録できる
  \item 今までに入力された成績情報を閲覧できる
  \item 名前か番号で検索できる
  \item 入力にエラーがある場合は通知できる
  \item 重複登録はできない
\end{itemize}

\section{基礎知識}



\section{実装方法}

\section{結果と考察}

\begin{thebibliography}{10}
  \bibitem{1}

  \url{}
\end{thebibliography}
\end{document}